\documentclass[12pt]{report}
\usepackage[utf8]{inputenc}
\usepackage{cite}
\usepackage[slovene]{babel}
\usepackage{titlesec}

\titleformat{\chapter}{\normalfont\huge}{\thechapter.}{20pt}{\huge}

\title{Izdelava sistema za upravljanje s programom lojalnosti Maestro}
\author{Ponudba št. SI-01-6458}
\date{December 2019}

\begin{document}

\maketitle

\chapter{Predstavitev podjetja}

Podjetje APTsystems je nastalo leta 2010, ko smo podjetje ustanovili nekdanji sošolci in študentje Fakultete za računalništvo in informatiko Univerze v Ljubljani. Predtem smo 10 let delovali na različnih področjih računalništva in informatike v podjetjih doma in v tujini.\\\\
V podjetju imamo 15 zaposlenih. Večina zaposlenih ima izobrazbo s področja računalništva in informatike, vsi imajo zaključeno najmanj 2. bolonjsko stopnjo (ali ekvivalent), med zaposlenimi sta tudi 2 doktorja znanosti. Zaposleni vodimo vse procese potrebne za razvoj dobre informacijske rešitve, saj se redno dodatno izobražujemo na domačih in mednarodnih konferencah, kar potrjujejo tudi certifikati, ki smo jih prejeli in jih ohranjamo.\\\\
V devetih letih od ustanovitve podjetja smo sodelovali že na več kot 15 projektih s slovenskimi in tujimi podjetji. Naš največji projekt do sedaj je bilo 3 letno sodelovanje z nemškim podjetjem Bosch, s katerim smo zasnovali linijo povezanih naprav za ti. pametne hiše, imenovano »Zmart«. Med drugim smo sodelovali tudi pri vzpostavitvi enotnega informacijskega sistema za vodenje plač, administracije in poslovne informatike podjetja Mahle Group, ki ima po celotnem svetu skoraj 80 000 zaposlenih, tudi več kot 2000 v Sloveniji. S podjetjem Mahle Group smo v letu 2016 zaključili projekt avtomatizirane tovarne v Šempetru pri Novi Gorici ter v letu 2017 podpisali dolgoročno sodelovanje na področju razvoja tehnologij modernizacije, avtomatizacije in izboljšanja produktivnosti industrijskih procesov - »Projekt Svetilnik«. Seznam vse zaključenih projektov lahko najdete na naši spletni strani\\\\
Podjetje je prejemnik mednarodne nagrade za kakovost IR podjetja Microsoft (leto 2015) ter Googlovega certifikata za kakovost (leto 2014, od takrat naprej ga vsa leta ohranjamo). Evropska komisija nam je v letu 2018 podelila značko »pionir na področju Industrije 4.0«, v letih 2014, 2016 in 2018 smo bili prejemniki Gazele za mala podjetja. Vsa leta od ustanovitve ohranjamo aktualne certifikate s področja razvoja IR in podatkovnih tehnologij podjetji Google, Microsoft, Amazon in Oracle (seznam vseh certifikatov in nagrad lahko najdete na naši spletni strani).\\\\

\section{Zakaj izbrati nas?}
Z vami, našo stranko, v začetku projekta dorečemo zahteve, ki jih končni izdelek mora izpolnjevati, nakar izberemo najprimernejše tehnologije za realizacijo projekta. Skozi sam projekt vas obveščamo o poteku in realizaciji projekta, večkrat pa vas bomo prosili tudi za vaše mnenje in vam predstavili delne izdelke, ki sovpadajo z časovnico projekta, ki jo zasnujemo skupaj z vami.\\\\
V podjetju vodimo projekte modularno, kar pomeni, da jih razdelimo na funkcionalne sklope pri čemer vsak sklop pripelje do nekega polizdelka. Pred pričetkom vsakega sklopa z vami podpišemo dogovor o zahtevah, časovnici in potrjevanju zahtev, ob zaključku sklopa podpišemo sklep o potrditvi in predaji izdelka tega sklopa.

\section{Izivi pri razvoju vaše IR}
Pri ponudbi, ki vam jo pošiljamo, smo upoštevali kompleksnost vaših zahtev. Kot največji izziv predstavlja ocena števila uporabnikov storitve. Zaradi velike količine uporabnikov, ki jo predvidevate, smo se odločili za drugačen princip razvoja aplikacije, ki je zahtevnejši in časovno potratnejši, vendar vam bo omogočal uporabo necentraliziranega sistema, ki ga lahko enostavno namestimo, kjer koli na področju vašega delovanja ter povežemo na obstoječe komponente sistema. Zasnovani sistem bo v osnovni različici lahko obdeloval do 20 000 000 zahtev na minuto, razširljiv pa je do 500 000 000 zahtev na minuto.\\\\
Pri vašem sistemu dajemo velik poudarek na varnosti, zato uporabljamo najvišje varnostne standarde, ki preprečujejo izpostavljanje kritičnih podatkov nepooblaščenim oseba, preprečujejo pa tudi do 99,9\% najpogostejših varnostnih udorov. \\\\
Velik del razvoja namenjamo testiranju sistema s čemer zagotavljamo 80\% »uptime« sistema, z redundanco sistema zagotavljamo 99,2\% »uptime« v prvih 5 letih delovanja sistema.

\pagebreak
\chapter{Predmet ponudbe}
Ponudba št. SI-01-6458 obsega razvoj celotnega informacijska sistema za upravljanje s programom lojalnosti, pri čemer vam priznamo celoten popust v postavkah, katere so že bile izdelane. Popust vam priznamo v postavkah 1.1, 1.2, 1.3, 1.4, 2.1, 2.3 in 2.4. Za več informacij glej preračun.

\section{Izračun končne cene}
V kolikor se izkaže, da je količina opravljenega dela v določenem sklopu manjša za več kot 10% od predvidenega, vam priznamo popust določen s podnjo tabelo:

\begin{center}
    \begin{tabular}{|c|c|}
        \hline
        Razlika med opravljenim in predvidenim delom (v \%) & Priznan popust (v \%)\\
        \hline
        10 - 25 & 10\\
        25 - 35 & 20\\
        35 - 45 & 30\\
        45 - 55 & 40\\
        $\geq$ 55 & 50\\
        \hline
    \end{tabular}
\end{center}

Pri čemer popust v enem sklopu ni višji od 50\% vrednosti sklopa, skupen popust pa ne presega več kot 30\% celotnega projekta.\\\\
Cena se lahko spreminja glede na druge pogoje, ki so del \textit{Dodatek 4}.

\section{Veljavnost ponudbe}
Ponudba je veljavna do 12. januarja 2020.

\pagebreak
\section{Predviden čas izdelave IR}
Predviden čas med začetkom specifikacije zahtev IR in predajo končnega sistema stranki je 30 tednov.\\\\
Predviden čas med začetkom načrtovanja IR in predajo končnega sistema stranki je 25 tednov.

\section{Plačilo}
Stranka je dolžna v osmih dneh od podpisa Dogovora o zahtevah, časovnici in potrjevanju zahtev, ki se podpiše pred začetkom vsakega sklopa, plačati akontacijo v vrednosti 50\% sklopa.\\\\
Stranka je dožna v osmih dneh po podpisu sklepa o potrditvi in predaji, ki se podpiše ob predaji izdelka vsakega sklopa, poravnati razliko med končnim zneskom sklopa in zneskom akontacije. \\\\
Izvajalec del v obeh primerih izda račun s preračunanim zneskom.\\\\\\\\\\\\\\\\\\\\\\\
Ponudbo pripravil: Jakob Marušič\\
Datum: 13.12.2019\\\\\\\\
Dodatki k ponudbi:
\begin{itemize}
    \item Predračun k ponudbi št. SI-01-6458
\end{itemize}

\end{document}