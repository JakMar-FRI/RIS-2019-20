\documentclass[a4paper,12pt]{report}
\usepackage[slovene]{babel}
\usepackage{listings}
\usepackage{graphicx}
\usepackage[section]{placeins}
\newcommand{\subtitle}[1]{%
  \posttitle{%
    \par\end{center}
    \begin{center}\large#1\end{center}
    \vskip0.5em}%
}

\lstset{
numberstyle=\small, 
numbersep=8pt, 
frame = single, 
language=SQL, 
framexleftmargin=15pt}

\begin{document}

\title{Skripta za pripravo na ustni izpit:\\Razvoj informacijskih sistemov}
\author{povzeto po predavanjih prof. Marka Bajca}

\maketitle

\chapter{Splošno o informacijskih sistemih}

\paragraph{Informacijski sistem} opredelimo kot množico medsebojno odvisnih komponent,
   med katere spadajo strojna in programska oprema ter ljudje, ki zbirajo, procesirajo,
   hranijo in porazdeljujejo podatke in s tem podpirajo delavne procese v organizaciji.
      

\paragraph{Delitev IS} Informacijske sisteme delimo na \textbf{formalne} in \textbf{neformalne}. 
Nefromalne IS ne moremo natančno definirati, saj ne vemo kako delujejo \textit{(npr. instinkt)}. Takšnih IS ne moremo sprogramirati.

\section{Vrste informacijskih sistemov}

   \paragraph{Transakcijski IS} pokriva vsako dnevne dogodke/transakcije (CDR - call data register).

   \paragraph{Upravljalski (poslovodski) IS} omogočajo planiranje \textit{(npr. ali se na študijskem programu izplača povečati vpis)}

   \paragraph{Direktorski IS}\mbox{}

   \paragraph{Odločitveni IS} omogočajo svetovanje za sprejemanje odločitev, bazirajo na določenem modelu \textit{(npr. odobritev kredita na banki)}

   \paragraph{Ekspertni IS} \textit{poskušajo} nadomestiti eksperta \textit{(npr. diagnostika pacienta na podlagi njegovih simptomov)}, osnovani so na \textbf{bazi znanja} (\textit{if/else} pravila)

   \paragraph{Sistemi za avtomatizacijo/ Sistemi za podporo delovni procesom}\mbox{}

   \subsection{Dellitev IS na ravni}
      \paragraph{Operativna raven} je najnižja raven, ki vključuje predvsem transakcijske sisteme, kjer hranimo veliko količino neagregiranih podatkov.
      Stopnja avtomatizacije je visoka, IS na tej ravni so visoko formalni.
      \paragraph{Taktična raven} vmesna raven, kjer so podatki zmerno agregirani, še vedno je prisotna visoka stopnja formalnosti in avtomatizacije \textit{(npr. beleženje porabljenih enot v mobilnem paketu)}
      \paragraph{Stratežka raven} je najvišja ravem na kateri so prisotni večinoma agregirani podatki. Takšni sistemi so povečini neformalni (ali zgolj delno formalni), avtomatizacije je malo ali ni prisotna.

   \section{Življenjski modeli razvoja informacijskih sistemov}

   \paragraph{Faze razvoja IS} Razvoj IS običajno delimo v naslednje faze:
      \begin{enumerate}
         \item analiza
         \item načrtovanje
         \item implementacija
         \item testiranje
         \item uvedba
         \item vzdrževanje
      \end{enumerate}
   \paragraph{Življenjski model razvoja} \textit{(SDLC - system development life cycle)} pove sosledje in način izvedbe faz v okviru razvoja IS.

   \pagebreak

   \subsection{Pristopi k razvoju IS}
   \subsubsection{Zaporedni ali slapovni pristop}
   Ko se ena faza konča, se druga začne.

   \begin{center}
      \begin{tabular}{|c|c|}
         \hline
         \textbf{Faza} & \textbf{Cilj}\\
         \hline
         analiza & zahteve\\
         načrtovanje & načrt\\
         izvedba & koda\\
         testiranje&\\
         uvedba&IS\\
         \hline
      \end{tabular}
   \end{center}

   \paragraph{Prednosti in slabosti takšnega pristopa}\mbox{}
   \begin{center}
      \begin{tabular}{|c||c|}
         \hline
         \textbf{Prednosti} & \textbf{Slabosti}\\
         \hline
         malo režijskega dela & prepozno testiranje\\
         & nenaravno\\
         \hline
      \end{tabular}
   \end{center}

   \subsubsection{Iterativni pristop}
   Faze razvoja so razdeljene v iteracije, v vsaki iteraciji se izvedejo vse faze. Vsaka iteracija ima nek končni izdelek, ki je del celotne končne rešitve.

   \begin{center}
      \begin{tabular}{|c||c|}
         \hline
         \textbf{Prednosti} & \textbf{Slabosti}\\
         \hline
         povratne informacije & težko načrtovanje iteracij\\
         \hline
      \end{tabular}
   \end{center}

   \paragraph{Planiranje iteracij na makro ravni - release planing} celoten razvoj razdelimo na makro iteracije, vsaka izmed iteracij ima kot končen izdelek nek del informacijske rešitve.
   Te iteracije imenujemo \textbf{release} in so omejene na 2-4 mesece. Pred zadnjim relesom moramo imeti izdelek, ki vsebuje vse minimalne zahteve projekta
   \textbf{MVP - minimal value product}.

   \paragraph{Planiranje iteracij na mikro ravni - iteration planing} Makro iteracije razdelimo na manjše iteracije, ki trajajo med enim in dvema tednoma.
   V tem času želimo razviti del rešitve, ki sestavlja končen izdelek, ki ga želimo razvit v tem releasu. Iteracije delimo še na \textbf{task-e}.

   \subsubsection{Prototipni pristop}
   Temelji na delnih prototipnih izdelkih. Analizo in razvoj prototipa opravimo v začetku v celoti in kot rezultat pridobimo delovni prototip.
   Ne \textbf{delovnem prototipu} potem izvajamo testiranje in uporabo, preko katerih se odločamo o nadaljnem razvoju in izboljšavah.

   \subsubsection{Inkrementalni pristop}
   Inkrementalni pristop nam omogoča, da projekt razdelimo na več samostojnih podproblemov. Vsak izmed podproblemov je samostojen del končne informacijske rešitve.
   Vsak del razvijamo posebej ter ga ob zaključku predamo stranki v uporabo ter nadaljujemo z razvojem naslednjega podproblema.

   \paragraph{MoSCoW} - \textit{must, should, could, won't}

   \begin{center}
      \begin{tabular}{|c||c|}
         \hline
         \textbf{Prednosti} & \textbf{Slabosti}\\
         \hline
         na začetku rešimo najbolj tvegane sklope & vseh IS ne moremo razdeliti\\& na samostojne enote\\
         \hline
      \end{tabular}
   \end{center}

   
   
   \section{Metodologije razvoja informacijskih sistemov}
   
   \paragraph{Metodologija} je zbirka filozofij, faz, postopkov, pravil, tehnik, orodij, dogovorov med sodelavci,... za razvijalce IS.

   \subsection{Zgodovina metodologij}
   \paragraph{do začetka 1970'} ni metodologij
   
   \paragraph{do zgodnjih '80} se začnejo pojavljati metodologije, ki dajejo poudarek na zajemu in analizi zahtev ter načrtovanju IR. 
   \\\\Pojavijo se tehnike podatkovnega in procesnega modeliranja.

   \paragraph{obdobje metodologije (do konca '90)} pojavi se velika količina novih metodologij, metodologije pridobijo veliko težo pri razvoju IR.

   \paragraph{obdobje ponovne ovenitve metodologij (danes)} začenja se obdobje v katerem težke metodologije zamenjajo lahke, metodologije
   postanejo odvisne in prilagojene potrebam programerja.

   \pagebreak

   \subsection{Formaliziranje metodologij}
   Metodologije imajo različno stopnjo formalnosti, bolj kot je metodologija formalizirana natančneje določena je. 
   Nedefinirane metodologije lahko povzročajo slabšo kvaliteto končnega izdelka, netransparentnost razvoja in težje vzdrževanje.

   Po drugi strani lahko omogočajo hitrejši, lažji in bolj prilagodljiv razvoj.

   \subsection{Vrste metodologij}
   Metodologije lahko delimo po več kriterijih:
   \begin{itemize}
      \item življenjski cikel
         \begin{itemize}
            \item zaporedni
            \item iterativni
            \item prototipni
         \end{itemize}
      \item tehnike
         \begin{itemize}
            \item objektna
            \item podatkovna
            \item procesna
            \item strukturna
         \end{itemize}
      \item teža metodologije
      \begin{itemize}
         \item obseg (število elementov, ki jih metodologija opisuje)
         \item gostota (zahtevan nivo formalnosti)
      \end{itemize}
      \item utežitev metodologije
      \begin{itemize}
         \item spredaj utežena - poudarek na analizi \& načrtovanju
         \item zadaj utežena - poudarek na kodiranju \& testiranju
         \item uravnotežena
      \end{itemize}
   \end{itemize}

   \paragraph{Komercialne metodologije}\mbox{}
   \begin{itemize}
      \item IE (information engineering)
      \item SSADM
      \item Rational Unified Process
      \item STRADIS
      \item agilne (lahke) metodologije
      \begin{itemize}
         \item XP - ekstremno programiranje
         \item SCRUM
         \item FDD
      \end{itemize}
   \end{itemize}

\chapter{Strukturni razvoj}

   \paragraph{Primer strukturne metodologije - informacijskih inženiring (IE)}\mbox{}
   Osnovne značilnosti IE:
   \begin{itemize}
      \item povezana množica tehnik planiranja, analize, načrtovanja, razvoja in vzdrževanja  IR
      \item pristop \textit{od zgoraj navzdol}
      \item avtomatizacija razvoja
      \item strateško planiranje
      \item povečana produktivnost
      \item predpostavlja: \textbf{poslovni sistemi so podatkovno, tehnični pa procesno ali dogodkovno usmerjeni}
      \item posebej obravnavamo podatke in posebej aktivnosti
   \end{itemize}
   IE delimo na 4 faze:
   \begin{enumerate}
      \item strateško planiranje (kaj?)
      \item analiza (kaj?)
      \item načrtovanje (kako?)
      \item izvedba (kako?)
   \end{enumerate}

   \pagebreak

   \section{Strateško planiranje}
   Strteško planiranje je proces izoblikovanja IS, ki organizaciji omogoča uresničitev ciljev in posredno zagotavlja konkurečno prednost.
   
   \paragraph{Cilj planiranja} je povezati razvoj IS s poslovno strategijo, načrtovanje pretoka informacij in procesov ter zmanjšati čas in stroške razvoja.

   \paragraph{Primer razdelitve dokumenta strateškega planiranja}\mbox{}
   \begin{enumerate}
      \item pregled in analiza stanja
      \item opredelitev poslovno-informacijske arhitekture
      \item opredelitev informacijske vizije
      \item opredelitev projektov
      \item izdelava akcijskega načrta
      \item spremljanje izvajanja in vzdrževanja strateškega plana
   \end{enumerate}

   \section{Zajem in specifikacija zahtev}
   Osnovni namen je opredeliti IR na način, ki bo omogočal:
   \begin{itemize}
      \item pri nakupu IR izbirati med obstoječimi rešitvami
      \item pri razvoju IR opredeliti osnovno funkcionalnost in druge nefunkcionalne zahteve ter omejitve za izgradnjo IR
   \end{itemize}

   Rezultat zajema je dokument (\underline{specifikacije}) z opredeljenimi funkcionalnostmi. Zajem zahtev opravi \underline{sistemski analitiki}
   v sodelovanju s \underline{poznavalci} \underline{problemske domene} oz. uporabniki.

   \paragraph{Osnovni koraki zajema}\mbox{}
   \begin{enumerate}
      \item zajem zahtev
      \item ureditev zahtev
      \item potrditev zahtev
   \end{enumerate}

   \subsection{Zajem zahtev}
   V zajemu zahtev uporabimo različne tehnike, od izbire le teh je odvisna kakovost specifikacije (razgovori, vprašalniki, 
   opazovanje dela, analiza obstoječega sistema,...).

   \paragraph{Napotki pri zajemu zahtev}
   \begin{itemize}
      \item objektivnost
      \item upoštevanje vseh možnosti
      \item posvečanje podrobnostim
      \item poudarek na novih in boljših rešitvah
      \item brez zadržkov pri zajemanju zahtev
   \end{itemize}

   \subsection{Ureditev zahtev}
   V tem koraku iz neformalno zapisanih zahtev oblikujemo formalno specifikacijo. \textbf{Specifikacija je dokument s formalno določenimi
   zahtevami, ki jih mora končni sistem zajemati.} Specifikacija je posredno določena z izbrano metodologijo, agilne metodologije določajo
   samo osnovne in okvirne zahteve, težke pa zahteve v celoti.\\

   
   Kvalitetna specifikacija je pravilna, celovita, nedvoumna, preverljiva, konsistentna in urejena po prioriteti/stabilnosti zahtev.
   \\
   
   Ključni vidiki zahtev obsegajo:
   \begin{itemize}
      \item funkcionalnost
      \item zmogljivost
      \item omejitve implementacije
      \item zunanje vmesnike
   \end{itemize}
   

   Zahteve delimo na \textbf{funkcionalne}, ki zajemajo želene funkcionalnosti sistemov, in 
   \textbf{nefunkcionalne}, ki se nanašajo na tehnične in druge nevsebinske zahteve.

   \paragraph{Običajna struktura specifikacije}
   \begin{enumerate}
      \item opis namena IR in njenega podsistema
      \item opis funkcionalnih zahtev
      \item opis nefunkcionalnih zahtev
      \item opis vmesnikov
      \item slovar pojmov
   \end{enumerate}

   \subsection{Potrditev zahtev}
   Specifikacijo potrdimo s strani naročnika preden nadaljujemo z nadaljnim razvojem IS.


   \section{Analiza}
   Osnovni namen analize je izdelati razumljiv opis poslovnega okolja, na katerega se nanaša IS.
   \\\\
   V analizi izdelamo \textbf{model sistema}, da na formalen način opredelimo potrebne podatkovne strukture in funkcije, ki
   te podatke uporabljajo.
   
   \paragraph{KAJ naj sistem podpira? (Rezultat analize)}
   \begin{itemize}
      \item model sistema
      \item predlog tehnične arhitekture
      \item prototip komponent uporabniškega vmesnika
      \item strategija testiranja
   \end{itemize}
   Postopke analize izvajajo \textbf{sistemski analitik, sistemski arhitekt} in ključni uporabnik.

   \subsection{Izdelava modela sistema}
   Pri strukturnem razvoju v času analize razvijemo 3 modele s katerimi si lažje viziualiziramo končni sistem
   \begin{enumerate}
      \item podatkovni modeli (konceptualni podatkovni model, entiteta-razmerje)
      \item procesni modeli (diagram razgradnje, diagram podatkovnih tokov, procesni diagrami)
      \item modeli procesne logike
   \end{enumerate}

   Modeli so izdelani v polformalnih tehnikah.

   \paragraph{Diagrami podatkovnih tokov} se fokusirajo na vhodne in izhodne podatke, ki jih potrebuje/proizvede nek proces.
   V modelu so vidni tokovi podatkov med podatkovnimi skladišči, funkcijami in zunanjimi sistemi, ne pa tudi sosledje dogodkov.

   \paragraph{Procesni diagram} prikazuje tok dogodkov ali potek določenega procesa.

   \paragraph{Model procesne logike} dopolnjuje procesni model, predvsem tiste procese, ki niso dovolj jasno definirani. Uporabljamo različne tehnike:
   \begin{itemize}
      \item strukturiran jezik
      \item odločitvene tabele
      \item odločitvena drevesa
      \item diagram prehajanja stanj
   \end{itemize}

   \subsection{Izdelava prototipov}
   Gre za neobvezen (ampak priporočljiv) korak, katerega cilj je prikazati izgled in osnovne funkcionalnosti sistema.

   \subsection{Izdelava predloge sistema (predlog tehnične arhitekture)}
   V okviru predloga definiramo komunikacijsko, programsko in strojno arhitekturo, ki je potrebna za vzpostavitev razvojnega, testnega in produkcijskega okolja.

   \paragraph{Delitev okolji} V času razvoja IR uporabljamo več vzporednih okolji (vsaj 2):
   \begin{enumerate}
      \item razvojno okolje
      \item testno okolje (ponovljivi testi)
      \item staging (kopija produkcije)
      \item produkcijsko
   \end{enumerate}

   \paragraph{Primer predloge}
   \begin{enumerate}
      \item Arhitektura sistema
      \begin{enumerate}
         \item ...
      \end{enumerate}
      \item Postopki, predpisi, standardi
      \begin{enumerate}
         \item ...
      \end{enumerate}
   \end{enumerate}

   \subsection{Opredelitev strategije testiranja}
   Opredelitev strategije testiranja je prva aktivnost v okviru testiranja.
   \begin{itemize}
      \item Kaj je predmet testiranja?
      \item Kdo bo testiral, kaj, kdaj in kako?
      \item Kje bo testno okolje?
      \item Struktura  testov
      \item ...
   \end{itemize}

   \subsection{Predstavitev rezultatov analize}
   Na koncu analize njene rezultate predstavimo končnemu uporabniku in nadaljujemo z načrtovanjem IS.

   \section{Načrtovanje}
   Glavni namen načrtovanja je izdelava načrta glede na specifikacije zbrane v analizi, načrt nam pove \textit{\textbf{kako}} sistem izdelamo.

   \paragraph{Cilj načrtovanja}
   \begin{itemize}
      \item izdelati načrt IR, ki ustreza analizi in upošteva tehnične omejitve
      \item dokumentirati specifikacije načrta tako, da omogoča nadaljno vzdrževanje sistema
      \item zasnovati strategijo prehoda na novo aplikacijo
   \end{itemize}

   \paragraph{Rezultati načrtovanja}
   \begin{itemize}
      \item načrt podatkovne baze
      \item načrt programskih modulov
      \item \textit{načrt dokumentacije}
      \item \textit{načrt testiranja}
      \item \textit{načrt namestitve in uvedbe}
   \end{itemize}
   Pri načrtovanju sodelujejo: načrtovalec podatkovne baze, načrtovalec aplikacije, skrbnik podatkovne baze, izdelovalec dokumentacije,
   uvajalec, poslovni lastnik in končni uporabnik.

   \subsection{Izdelava načrta podatkovne baze}
   Iz procesnega modela, ki smo ga pripravili v analizi, oblikujemo logični in/ali fizični model podatkovne baze, ki ustreza tehnični arhitekturi.

   \subsection{Izdelava načrta programskih modulov}
   Namen je prikazati kako bodo bili, v analizi identificirani, procesi in funkcije v končni IR prikazani.

   \subsection{Izdelava načrta dokumentacije}
   Z aktivnostjo želimo določiti obseg in strukturo dokumentacije ter izbiro standardov in vzorcev. Dokumentacijo delimo na:
   \begin{itemize}
      \item uporabniško dokumentacijo
      \item sistemsko-tehnično dokumentacijo
      \begin{itemize}
         \item podatkovni model
         \item arhitektura sistema
         \item komponente sistema
         \item opis testnega, razvojnega in produkcijskega okolja
         \item ...
      \end{itemize}
      \item navodila za operativno skrbništvo
      \begin{itemize}
         \item izdelava varnostnih kopij
         \item izklop sistema
         \item posodabljanje sistema
         \item ...
      \end{itemize}
   \end{itemize}

   \subsection{Izdelava načrta testiranja}
   Najpomebnejša lastnost testov je njihova \textbf{ponovljivost}, v sklopu načrtovanja pripravimo načrt testiranja, ki opredeljuje:
   \begin{itemize}
      \item kdo testira
      \item kaj testira
      \item kako testira
      \item kdaj in zakaj testira
   \end{itemize}

   \subsection{Izdelava načrta namestitve in uvedbe}
   Načrt namestitve in uvedbe definira uvedbo v testno in producijsko okolje, ter med drugim tudi kdo bo kaj tetsiral 
   (v razvoju razvojna ekipa, v testnem in produkcijskem pa tudi končni uporabniki).
   \\\\
   Načrt vsebuje:
   \begin{itemize}
      \item načrt namestitve
      \item načrt dodelitve pravic
      \item načrt prevedbe podatkov (iz starega v nov sistem)
      \item načrt uvajanja
      \item načrt za izvedbo prevzemnega in končnega testa
      \item načrt prehoda na nov sistem
   \end{itemize}

   \subsection{Testiranje}
   S testiranjem želimo doseči preverjeno delujočo aplikacijo. V okviru razvoja IR se različna testiranja izvajajo različno pogosto in ob različnih časih.
   Teste delimo na:
   \begin{itemize}
      \item teste programskih enot
      \item teste integracije
      \item sistemske teste
      \item teste sprejemljivosti
   \end{itemize}

   Izvajalci testov so razvijalci (predvsem v razvojnem okolju), preizkuševalci (sistematično testiranje v testnem okolju) in
   končni uporabniki (testno in produkcijsko okolje).


   \section{Namestitev in uvedba}
   \subsection{Prehod na nov sistem}
   
   \paragraph{Fazni pristop} star sistem nadomestimo z novim v več korakih, pri čemer oba sistema delujeta medseboj. Potreben je razvoj
   vmesnikov, ki omogočajo komunikacijo med sistemoma. Postopno lahko sistem menjujemo po področjih, lokacijah, modulih,...

   \paragraph{Zamenjava ali vse naenkrat} velik riziko, saj star sistem ne deluje več, novega pa v produkcijskem okolju še nismo testirali.

   \paragraph{Vzporedno delovanje} starega in novega sistema. Je najbolj varen način, vendar podvaja delo končnemu uporabniku.


\chapter{Objektni razvoj}

   Glavne razlike v primerjavi s strukturnim razvojem so pri analizi, načrtovanju in izvedbi.

   \paragraph{Objekt} lahko predstavlja fizično entiteto ali koncpetualni pojem. S stališča razvoja IR je objekt koncept,
   abstrakcija z natančno določenimi mejami in lastnostmi, ki so pomemben za IR. Objekt ima:
      \begin{itemize}
         \item stanje (objekt se zaveda svojega stanja - lahko \textit{pove} svoje stanje)
         \item obnašanje (funkcije, ki jih objekt \textit{dela} in kar se z njim da delati \textit{"pisalo piše" - kdo piše?"})
         \item entiteto
      \end{itemize}
   Odnos med objektom in razredom je podobno odnosu med entiteto in entitetnim tipom.

   \paragraph{Enkapsulacija ali skrivanje podatkov}
   O objektu ne rabimo vedeti vsega, ampak samo tisto, kar potrebujemo za njegovo uporabo (vmesnik). 
   Enkapsulacija zahteva, da skrijemo:
      \begin{itemize}
         \item implementacijo obnašanja, ki je na voljo prek vmesnika
         \item podatke znotraj objekta, ki so potrebni za implementacijo obnašanja in beležijo stanje objekta v trenutku obstoja
      \end{itemize}

      \paragraph{Dedovanje in hierhija} Dedovanje uporabimo, da nekemu razredu dodelimo lastnosti nekega drugega, dedovanega razreda.
      \textit{Npr. študent in delavec oba dedeujeta lastnosti kot so ime in priimek od razreda oseba.}

      \paragraph{UML} je standariziran nabor tehnik za modeliranje objektnih sistemov, med UML diagrame umeščamo: razredne, komponetne, use-case diagrame, diagrame stanj,\dots

      \section{Osnove procesa objektnega razvoja RUP}
      \textbf{RUP} rational unified process (proces določa kdo, kdaj, kaj dela)

      \paragraph{Glavne značilnosti RUP}
         \begin{itemize}
            \item snernice za učinkovit razvoj
            \item zmanjšuje tveganje, povečuje predvidljivost
            \item zajema in vpeljuje najboljše izkušnje (učenje iz izkušenj,\dots)
            \item pospešuje vizijo
            \item vpeljuje načrt za vpeljavo
         \end{itemize}

      \paragraph{6 najboljših praks v razvoju IR, ki jih vpeljuje RUP}
         \begin{center}
            \begin{tabular}{|c|}
               \hline
               iterativen razvoj\\
               \hline
               \begin{tabular}{c|c|c|c}
                  obvladovanje & uporaba  & virtualno & preverjanje\\
                  zahtev & komponentne arhitekture & modeliranje & kakovosti\\
               \end{tabular}\\
               \hline
               nadzorovanje sprememb\\
               \hline
            \end{tabular}
         \end{center}

      \paragraph{Arhitekrura IR} določa smernice razvoja, zgradbo izdelka in sestavo skupine.
      RUP daje velik poudarek na arhitekturo že v začetnih iteracijah zato zahteva 4+1 pogled:
         \begin{itemize}
            \item logični pogled (analitik/razvijalec) - zgradba sistema
            \item izvedbeni pogled (programer) - upravljanje
            \item postavitveni pogled (sistemski inženir) - topologija sistema
            \item procesni pogled (sistemski poznavalec)
            \item primer uporabe (končni uporabnik) - funkcionalnost
         \end{itemize}
      
      \paragraph{Kaj še omogoča RUP?}
         \begin{itemize}
            \item vzpostavitev in ohranitev nadzora nad projektom
            \item obvladovanje kompleksnosti
            \item vzdrževanje celovitosti sistema
            \item razširanje možnosti ponovne uporabe
            \item pospeševanje komponentno usmerjen razvoja
         \end{itemize}

      \subsection{RUP faze}
      \begin{itemize}
         \item začetna faza - vzpostavitev sistema, opredelitev okvirjev in načrtovanje virov
         \item zbiranje informacij - specifikacija zahtev, načrtovanje virov
         \item konstrukcija - razvoj sistema
         \item prevzem - predaja izdelka
      \end{itemize}

      \section{Zajem zahtev}
      Namen zajema zahtev je doseči soglasje s stranko oz. uporabnikom, kaj naj sistem dela. Na drugi strani omogočimo
      razvijalcem boljše razumevanje zahtev sistema ter določimo meje sistema, ki nam zagotavljajo osnovo za načrtovanje tehnične vsebine.

      RUP, za razliko od strukturnega razvoja, vpeljuje višjo stopnjo formalnosti:
      \begin{itemize}
         \item primeri uporabe
         \item opisi s tokovi dogodkov
         \item uporaba drugih tehnik UML v posebnih primerih (diagrami stanj, diagrami aktivnosti, diagrami iterakcije)
      \end{itemize}

      \paragraph{Diagram primerov uporabe} prikazuje primere uporabe ter akterje, ki primere uporabe uporabljajo.
      Znotraj posameznih primerov uporabe določimo še tok dogodkov (vsak tok dogodkov ima osnovni in enega ali več alternativnih tokov),
      ki opisuje zaporedje izvajanja posameznih dogodkov.
      \\\\
      Pri izdelavi modela uporabe najprej definiramo akterje ter primere uporabe, ki jih med seboj povežemo. V primeru odvečnih
      primerov uporabe, akterjev ali povezav jih pri pregledu odstranimo.
      \\\\
      Poleg primerov uporabe in \textbf{opisov primerov uporabe} v fazi analize definiramo še \textbf{slovar} in \textbf{dodatne specifikacije}, 
      ki zajemajo funkcionalnosti, uporabnost, zanesljivost, učinkovitost, podporo in omejitve pri načrtovanju.

      \section{Objekta analize in načrtovanja}
      \paragraph{Primerjava faz analize in načrtovanja}
      \begin{center}
         \begin{tabular}{|c|c|}
            \hline
            \textbf{Analiza} & \textbf{Načrtovanje} \\
            \hline
            razumevanje problema & razumevanje rešitve\\
            idealiziramo načrtovanje & operacije in atributi\\
            obnašanje & zmogljivost\\
            struktura sistema & blizu programski kodi\\
            funkcijske zahteve & nefunkcionalne zahteve\\
            majhen model & velik model \\
            & življenjski cikel objektov\\
            \hline
         \end{tabular}
      \end{center}

   \subsection{Analiza arhitekture}
   \paragraph{Identifikacija ključnih abstrakcij} skozi modele iz analize in zajema zahtev izluščimo najpomembnejše (ključne) objekte (abstrakcije).

   \paragraph{Analiza potreb po sistemskih virih} kot so trajnost, procesna komunikacija, porazdeljevanje, obvladovanje transakcij, varnost,\dots

   \paragraph{Opredelitev arhitekturne ravni} predstavljajo aplikacijo z različnih nivojev abstrakcije, primeri:
      \begin{itemize}
         \item model-view-controller
         \item pipes and filters
         \item blackboard
      \end{itemize}
   Aplikacija lahko uporablja več arhitekturnih vzorcev. Organiziranost arhitekturnih ravni pokažemo s paketi, ti so
   elementi sistema, ki lahko vključujejo druge elemente modela.


   \subsection{Analiza primerov uporabe}
   \paragraph{Dopolnitev opisov primerov uporabe} primere uporabe dopolnimo s podatki, ki so potrebni za nadaljni razvoj.

   \paragraph{Identifikacija potrebnih razredov za realizacijo posameznih primerov uporabe} 
   Razrede v osnovi delimo na kontrolne (en primer uporabe ima navadno enega), mejne (za vsak zunanji sistem imamo enega)
   in entitetne (poslovni razredi s katerimi hranimo in upravljamo podatke). 
   Med razredi definiramo tudi povezave (te so lahko neusmerjene ali usmerjene), ki jih poimenujemo in določimo števnost.

   \paragraph{Razdelitev odgovornosti posameznim razredom}
   Z uporabo diagrama zaporedja iz primera uporabe definiramo kateri razred izvede katero izmed aktivnosti. 
   Mejni razredi prevzamejo odgovornosti komunikacije z akterji, poslovni podatke, kontrolni pa (običajno) pomembnejši del toka dogodkov.

   \paragraph{Podrobna analiza potrebnih sistemskih virov} Sestavimo seznam vseh potrebnih sistemskih storitev ter razredov, ki potrebujejo sistemske storitve.
   V tem delu tudi opišemo lastnosti sistemskih storitev.

   \paragraph{Poenotenje razredov pridobljenih z analizo}

\end{document}